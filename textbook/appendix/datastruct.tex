\chapter{Background on Data Structures}
\label{ch:app:datastruct}

We assume that the reader is familiar with basic data structures such
as arrays and with the basic data types built in to most programming
languages (such as integer, floating point, string, etc). Many programming
applications require the programmer to create complicated combinations
of the built-in structures. Some languages make this easy by allowing
the user to define new data types (for example Java or C++ classes), and
others do not (for example C, Fortran). These new data types are concrete
implementations in the given language of \defnfont{abstract data type}s
(\defnfont{ADT}s), which are mathematically specified.

\section{Informal discussion of ADTs}
\label{sec:app:adt-informal}

An ADT consists of a set with certain operations on it. How those
operations are to be carried out is not our concern here. It is up to the
programmer to choose an implementation that suits the given application.

Some of the key ADTs are: list, stack, queue, priority queue, dictionary,
disjoint sets. We discuss them each below in turn, semi-formally. 

A \defnfont{container} is a collection of objects from some universal
set $U$. Basic operations are to create a new empty container, insert an
element, check whether the container is empty (the \algfont{isEmpty} operation). 
We assume that each element when inserted returns a locator that identifies its 
position uniquely.

We can then try to \algfont{find} an element. Depending on the
additional operations defined, this may be easy or difficult. We may
need to enumerate all locators. We can also find the \algfont{size}
of a container by enumerating all locators. Again, this may be very
inefficient, and for certain applications a special size operation
may be defined. Similarly, many container ADTs have a 
\algfont{delete} operation. Some of these allow quick removal of an
element, while others have to find it first, which may be slower.

Other operations can be expressed in terms of the basic ones. For
example, we can sometimes modify or update an element by finding it,
remembering its locator, deleting it, then inserting the new value at
the given location. This procedure is sometimes very inefficient, so a
special update operation may be required in some situations. We normally
try to have as few basic operations as possible, and other operations
such as sorting are expressed in terms of these (this is the aim of
``generic programming").

As we see a container is very general. Some of the important container
ADTs are listed below.

A \defnfont{list} is a container that stores elements in a linear
sequence. Some basic operations are to \algfont{insert} in a given
position in the sequence, to \algfont{delete} an element at a given
position. To find an element requires sequential search, enumerating
all locators until the element is found or we run out of locators. The first element 
of a list is called the \defnfont{head} and the last is the \defnfont{tail}.
A \defnfont{sublist} is a contiguous piece of the list, that can be traversed by the 
iterator with no gaps. If we divide the list into two sublists by choosing an 
element $x$ and letting the head sublist consist of all elements before $x$, and the 
tail sublist consist of all elements after it (either sublist, or neither, may 
contain $x$, depending on the situation).

The main data structures used for implementing a list are arrays and
singly- and doubly-linked lists. They have different properties. For
example, to find the middle element of a list implemented as an array
is a constant-time operation but this is not true for the linked lists,
since one must traverse the list, not having direct access to the middle
element as in an array. On the other hand, to insert an element in a
given (nonempty) position in an array takes longer than with the linked
list implementation.

A \defnfont{sorted list} is a container of elements from a totally
ordered set $U$, with the same basic operations as a list, except that
the elements are kept sorted in ascending order.

A \defnfont{stack} is a restricted kind of list in which insertion and deletion
occur at the same end (``top") of the list, and at no other position. The
basic operations are \algfont{insert} (also called \algfont{push}),
\algfont{delete} (also called \algfont{pop}), and \algfont{getTop}
(also called \algfont{peek}) which returns the element at the top of the
list without removing it.

A \defnfont{queue} is a restricted kind of list in which insertion occurs at one end
(the ``tail") and deletion occurs at the other end (the ``head"). The
basic operations are \algfont{insert} (also called \algfont{enqueue} or 
\algfont{push}), \algfont{delete} (also called \algfont{dequeue} or 
\algfont{pop}), and \algfont{getHead} (also called \algfont{peek}) which 
returns the element at the head without removing it.

A \defnfont{priority queue} is a container of elements from a totally
ordered set $U$ that allows us to \algfont{insert} an element, to
find the smallest element with \algfont{peek}, and to remove the
smallest element with \algfont{delete}.
 A more advanced operation is \algfont{decreaseKey} which finds and makes 
an element smaller. A general delete operation is not usually defined.

A priority queue can be well implemented using a binary heap, if the
operation \algfont{decreaseKey} is not required to be particularly efficient. 
Otherwise, more sophisticated implementations are normally used.

A \defnfont{dictionary} (also called \defnfont{table}, 
\defnfont{associative array} 
or \defnfont{map}) is a a container with basic operations
\algfont{find}, \algfont{insert}, and \algfont{delete}. It is usually
also desired to perform an \algfont{update} operation many times, since
dictionaries are often used for dynamic databases.

Some of these ADTs can be used to simulate other ones. For example,
a dictionary can be implemented using a list. Insertion occurs
after the last element, and finding is via sequential search. For
practical situations where a dictionary is used, the \algfont{find}
operation is used a lot, so such an inefficient implementation would not
normally be used. Similarly, one can use a list to simulate a priority
queue. Insertion occurs at one end and finding the maximum element by
sequential search. Or we could use a sorted list, where insertion occurs
at the correct point and removing the minimum element is particularly
easy, since it is just removing the first element.

\section{Notes on a more formal approach}
\label{sec:app:adt-formal}

The discussion above still doesn't define the various ADTs in a completely
satisfactory way. As in abstract algebra, we must specify not only the
basic operations but also their properties using a set of axioms. It can
be quite difficult to do this succinctly. Also, whereas in the case of
algebra the basic structures (group, ring, field \dots) have been agreed
on for many decades, the axiomatic definitions of the main ADTs do not
seem to be completely standardized yet. So we shall not give a complete
axiomatic presentation, but limit ourselves to an example.

The stack ADT could be defined as follows.

A stack on a set $U$ is a set $S$ with operations \algfont{push},
\algfont{pop}, \algfont{peek}, \algfont{isEmpty}. There is an empty
stack called $\varepsilon$ which corresponds to $S$ being the empty
set. These operations do the following: \algfont{push} takes an ordered
pair consisting of a stack and an element of $U$ as an argument, and
returns a stack; \algfont{pop} takes a stack as an argument and returns
an element of $U$; \algfont{peek} takes a stack as an argument and returns
either an element of $u$ or ``ERROR"; \algfont{isEmpty} returns either
0 (false) or 1 (true). The axioms for every stack S and element $x$ of $U$ are 
as follows.
\begin{itemize}
\item \algfont{isEmpty}($\varepsilon$) = 1
\item \algfont{pop}(\algfont{push}($S$, $x$)) = $S$
\item \algfont{peek}(\algfont{push}($S$, $x$)) = $x$
\end{itemize}

